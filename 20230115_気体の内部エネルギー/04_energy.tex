\hypertarget{ux6c17ux4f53ux306eux5185ux90e8ux30a8ux30cdux30ebux30aeux30fcux3068ux7d76ux5bfeux6e29ux5ea6ux304cux6bd4ux4f8bux95a2ux4fc2ux306bux3042ux308bux3053ux3068ux306eux8a3cux660e}{%
\section{気体の内部エネルギーと絶対温度が比例関係にあることの証明}\label{ux6c17ux4f53ux306eux5185ux90e8ux30a8ux30cdux30ebux30aeux30fcux3068ux7d76ux5bfeux6e29ux5ea6ux304cux6bd4ux4f8bux95a2ux4fc2ux306bux3042ux308bux3053ux3068ux306eux8a3cux660e}}

気体の内部エネルギーは熱運動の運動エネルギーと分子間力によるエネルギーからなるが,分子同士は互いに十分に離れているため,後者は無視できる.
つまり,気体の内部エネルギーを求めるには,構成する分子の運動エネルギーを求めれば良い.

粒子(質量\(m\),
速度\(v\))が壁面(面積\(S\))に完全弾性衝突で衝突したときの力積を考える.

壁面が粒子から受けた力積の\(x\)成分の大きさ\(I_x\)は,

\({I_x}=2m{v_x}\)

で表せる.

壁面を底面とする高さ\(L\)の容器において,壁面に衝突する回数\(T_x\)は

\({T_x} = \frac{粒子のx方向への速さ}{壁面に戻ってくるまでのx方向の道のり} = \frac{v_x}{2L}\)

で表せる.

よって,「壁面が1つの粒子から受ける力積\(I_1\)」は,

\({I_1}={I_x}\cdot{T_x}=2{m}{v_x}\cdot\frac{v_x}{2L}={m}\cdot\frac{{{v_x}^2}}{L}\)

となる.

\(n\)molの粒子の個数を\(N\)とすると(\(N_A\):アボガドロ数)

\(N=n{N_A}\)

なので,「壁面が\(n\)molの粒子から受ける力積\(I_n\)」は,

\({I_n}={N}\cdot{I_1}=(n{N_A})\cdot({m}\cdot\frac{{{v_x}^2}}{L})=\frac{m{{v_x}^2}}{L}\cdot{n{N_A}}\)

この力積\(I_n\)は,「\(n\)molの粒子からなる気体が壁面を押す力\(F\)」と同義である.
\(圧力=\frac{力の大きさ}{力が加わる面積}\)であることを踏まえると,「気体の圧力\(p\)」は

\(p=\frac{F}{S}=(\frac{m{{v_x}^2}}{L}\cdot{n{N_A}})\cdot\frac{1}{S}=\frac{m{{v_x}^2}n{N_A}}{LS}\)

と表せる.

ここまでは\(x\)方向についてのみ考えたが,これらの反応は\(y, z\)方向についてもそれぞれ成立しているはずである.

気体分子の速度\(v\)の二乗平均\(\overline{v^2}\)は,

\(\overline{v^2}=\overline{{v_x}^2}+\overline{{v_y}^2}+\overline{{v_z}^2}\)

で表せ,かつ,その等方性により

\(\overline{{v_x}^2}=\overline{{v_y}^2}=\overline{{v_z}^2}\)

であるため,

\(\overline{{v_x}^2}=\overline{{v_y}^2}=\overline{{v_z}^2}=\frac{1}{3}\overline{v^2}\)

となる.また,粒子が飛び回る容器(底面\(S\),
高さ\(L\))の体積を\(V(={S}\cdot{L})\)とすると「気体の圧力\(p\)」は

\(p=\frac{m{{v_x}^2}n{N_A}}{LS}=\frac{m({\frac{1}{3}\overline{v^2}})n{N_A}}{(V)}=\frac{m\overline{v^2}n{N_A}}{3V}\)

となる.これを運動エネルギーの式\(K=\frac{1}{2}m{v}^2\)を参考に変形すると,

\(3pV=m\overline{v^2}n{N_A}\)

\(\frac{3pV}{2n{N_A}}=\frac{1}{2}m\overline{v^2}\)

となる.
気体の内部エネルギー\(U\)は,構成する分子の運動エネルギー\(K\)であるため,

\(U=K=\frac{1}{2}m\overline{v^2}=\frac{3}{2}\cdot\frac{pV}{n{N_A}}\)

となる.ボイル-シャルルの法則から導出される理想気体の状態方程式\(pV=nRT\)が適用される範囲においては,

\(U=\frac{3}{2}\cdot\frac{pV}{n{N_A}}=\frac{3}{2}\cdot\frac{nRT}{n{N_A}}=\frac{3}{2}\cdot\frac{R}{N_A}\cdot{T}\)

と式変形ができる.このとき,\(R, N_A\)はいずれも\(気体定数\)と\(アボガドロ数\)で定数\footnote{\(\frac{R}{N_A}\)をボルツマン定数\(k(あるいは{k_b})\)とすることが多い}のため,理想気体における内部エネルギーは絶対温度に比例する\footnote{粒子の運動エネルギーから導出した式に理想気体の状態方程式を加えることで,気体の内部エネルギーと絶対温度との比例関係を見出せた}.
